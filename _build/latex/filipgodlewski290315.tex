%% Generated by Sphinx.
\def\sphinxdocclass{report}
\documentclass[letterpaper,10pt,polish]{sphinxmanual}
\ifdefined\pdfpxdimen
   \let\sphinxpxdimen\pdfpxdimen\else\newdimen\sphinxpxdimen
\fi \sphinxpxdimen=.75bp\relax
\ifdefined\pdfimageresolution
    \pdfimageresolution= \numexpr \dimexpr1in\relax/\sphinxpxdimen\relax
\fi
%% let collapsible pdf bookmarks panel have high depth per default
\PassOptionsToPackage{bookmarksdepth=5}{hyperref}

\PassOptionsToPackage{booktabs}{sphinx}
\PassOptionsToPackage{colorrows}{sphinx}

\PassOptionsToPackage{warn}{textcomp}
\usepackage[utf8]{inputenc}
\ifdefined\DeclareUnicodeCharacter
% support both utf8 and utf8x syntaxes
  \ifdefined\DeclareUnicodeCharacterAsOptional
    \def\sphinxDUC#1{\DeclareUnicodeCharacter{"#1}}
  \else
    \let\sphinxDUC\DeclareUnicodeCharacter
  \fi
  \sphinxDUC{00A0}{\nobreakspace}
  \sphinxDUC{2500}{\sphinxunichar{2500}}
  \sphinxDUC{2502}{\sphinxunichar{2502}}
  \sphinxDUC{2514}{\sphinxunichar{2514}}
  \sphinxDUC{251C}{\sphinxunichar{251C}}
  \sphinxDUC{2572}{\textbackslash}
\fi
\usepackage{cmap}
\usepackage[T1]{fontenc}
\usepackage{amsmath,amssymb,amstext}
\usepackage{babel}



\usepackage{tgtermes}
\usepackage{tgheros}
\renewcommand{\ttdefault}{txtt}



\usepackage[Sonny]{fncychap}
\ChNameVar{\Large\normalfont\sffamily}
\ChTitleVar{\Large\normalfont\sffamily}
\usepackage{sphinx}

\fvset{fontsize=auto}
\usepackage{geometry}


% Include hyperref last.
\usepackage{hyperref}
% Fix anchor placement for figures with captions.
\usepackage{hypcap}% it must be loaded after hyperref.
% Set up styles of URL: it should be placed after hyperref.
\urlstyle{same}

\addto\captionspolish{\renewcommand{\contentsname}{Contents:}}

\usepackage{sphinxmessages}
\setcounter{tocdepth}{1}



\title{FilipGodlewski290315}
\date{10 lis 2025}
\release{1.0}
\author{Filip Godlewski}
\newcommand{\sphinxlogo}{\vbox{}}
\renewcommand{\releasename}{Wydanie}
\makeindex
\begin{document}

\ifdefined\shorthandoff
  \ifnum\catcode`\=\string=\active\shorthandoff{=}\fi
  \ifnum\catcode`\"=\active\shorthandoff{"}\fi
\fi

\pagestyle{empty}
\sphinxmaketitle
\pagestyle{plain}
\sphinxtableofcontents
\pagestyle{normal}
\phantomsection\label{\detokenize{index::doc}}


\sphinxAtStartPar
Add your content using \sphinxcode{\sphinxupquote{reStructuredText}} syntax. See the
\sphinxhref{https://www.sphinx-doc.org/en/master/usage/restructuredtext/index.html}{reStructuredText}
documentation for details.

\sphinxstepscope


\chapter{Wprowadzenie do technologii informacyjnych}
\label{\detokenize{rozdzial1/index:wprowadzenie-do-technologii-informacyjnych}}\label{\detokenize{rozdzial1/index::doc}}
\sphinxAtStartPar
Technologie informacyjne to zbiór narzędzi, metod i procesów, które
umożliwiają przetwarzanie, gromadzenie, przesyłanie i wykorzystywanie
informacji w sposób efektywny. Ich rozwój w ostatnich dekadach
diametralnie zmienił sposób, w jaki funkcjonują społeczeństwa,
gospodarki oraz instytucje publiczne. Współczesny świat praktycznie nie
może istnieć bez technologii informacyjnych, ponieważ są one podstawą
niemal każdego sektora życia. Rozwój technologii informacyjnych nie
tylko usprawnia komunikację i dostęp do danych, ale także kształtuje
nowe modele biznesowe, edukacyjne oraz administracyjne. W tym obszarze
powstają innowacyjne rozwiązania, które wspierają inteligentne miasta,
zrównoważony rozwój oraz poprawę jakości życia w społeczeństwach.

\sphinxstepscope


\chapter{Kluczowe obszary zastosowań technologii informacyjnych}
\label{\detokenize{rozdzial2/index:kluczowe-obszary-zastosowan-technologii-informacyjnych}}\label{\detokenize{rozdzial2/index::doc}}
\sphinxAtStartPar
Technologie informacyjne obejmują wiele dziedzin, z których każda
pełni inną rolę w procesie cyfrowej transformacji. Najważniejsze z
nich to:
\begin{enumerate}
\sphinxsetlistlabels{\arabic}{enumi}{enumii}{}{.}%
\item {} 
\sphinxAtStartPar
Infrastruktura IT \textendash{} sprzęt komputerowy, serwery, sieci i urządzenia
mobilne, które tworzą fundament systemów informatycznych. Obejmuje
także centra danych, serwery w chmurze, urządzenia IoT i sprzęt do
analizy danych w czasie rzeczywistym.

\item {} 
\sphinxAtStartPar
Oprogramowanie \textendash{} systemy operacyjne, aplikacje użytkowe, systemy
zarządzania bazami danych, które pozwalają użytkownikom wykonywać
konkretne zadania i analizować dane. Oprogramowanie staje się coraz
bardziej inteligentne dzięki integracji algorytmów uczenia
maszynowego i sztucznej inteligencji.

\item {} 
\sphinxAtStartPar
Usługi sieciowe i chmurowe \textendash{} umożliwiają przechowywanie, przesyłanie
i przetwarzanie danych w środowisku online, oferując skalowalność i
elastyczność. Popularne modele to Software as a Service (SaaS),
Platform as a Service (PaaS) i Infrastructure as a Service (IaaS).

\item {} 
\sphinxAtStartPar
Cyberbezpieczeństwo \textendash{} obejmuje metody ochrony danych i systemów przed
atakami, włamaniami, a także przed nieautoryzowanym dostępem. W tym
obszarze szczególnie istotne są technologie szyfrowania,
uwierzytelniania wielopoziomowego oraz monitoring zagrożeń.

\item {} 
\sphinxAtStartPar
Analiza danych i sztuczna inteligencja \textendash{} umożliwia przetwarzanie
ogromnych ilości danych, wykrywanie wzorców, przewidywanie trendów i
automatyzację procesów. Narzędzia analityczne wspierają decyzje
biznesowe, prognozowanie rynków oraz personalizację usług.

\item {} 
\sphinxAtStartPar
Rozwój interfejsów użytkownika \textendash{} projektowanie intuicyjnych i
dostępnych systemów, które ułatwiają interakcję człowieka z
technologią. Dotyczy to zarówno aplikacji mobilnych, jak i
rozbudowanych systemów korporacyjnych.

\item {} 
\sphinxAtStartPar
Telekomunikacja i łączność bezprzewodowa \textendash{} podstawy działania sieci
komórkowych, Wi\sphinxhyphen{}Fi, sieci 5G oraz przyszłych technologii 6G, które
umożliwiają natychmiastowy dostęp do informacji i komunikację w
czasie rzeczywistym.

\end{enumerate}

\sphinxAtStartPar
Każda z tych dziedzin jest rozwijana równolegle na całym świecie, co
prowadzi do powstawania innowacyjnych rozwiązań i zwiększa złożoność
całego ekosystemu technologii informacyjnych.

\sphinxstepscope


\chapter{Etapy rozwoju technologii informacyjnych}
\label{\detokenize{rozdzial3/index:etapy-rozwoju-technologii-informacyjnych}}\label{\detokenize{rozdzial3/index::doc}}
\sphinxAtStartPar
Rozwój technologii informacyjnych można podzielić na kilka kluczowych
etapów, z których każdy wnosi nowe możliwości i wyzwania:
\begin{itemize}
\item {} 
\sphinxAtStartPar
Era mechaniczna \textendash{} pierwsze urządzenia liczące, takie jak abakus,
mechaniczne kalkulatory i maszyny Babbage’a. Był to początek
automatyzacji obliczeń i wprowadzenie systematycznego podejścia do
przetwarzania danych.

\item {} 
\sphinxAtStartPar
Era elektroniczna \textendash{} pojawienie się komputerów opartych na układach
scalonych i tranzystorach. Rozwój systemów operacyjnych umożliwił
efektywne zarządzanie zasobami sprzętowymi oraz tworzenie aplikacji
do obliczeń naukowych i biznesowych.

\item {} 
\sphinxAtStartPar
Era sieciowa \textendash{} rozwój Internetu, sieci komputerowych i globalnej
komunikacji, które zrewolucjonizowały wymianę informacji na świecie.
W tym okresie powstały pierwsze przeglądarki, poczta elektroniczna i
protokoły sieciowe.

\item {} 
\sphinxAtStartPar
Era mobilna \textendash{} dominacja urządzeń przenośnych, smartfonów, tabletów i
technologii bezprzewodowych, umożliwiających dostęp do informacji w
czasie rzeczywistym oraz rozwój aplikacji mobilnych i platform
społecznościowych.

\item {} 
\sphinxAtStartPar
Era sztucznej inteligencji \textendash{} integracja systemów uczących się i
automatyzacja procesów decyzyjnych, rozwój chatbotów, asystentów
głosowych oraz systemów analitycznych w przedsiębiorstwach. AI
wspiera również procesy predykcyjne, automatyczne diagnozy medyczne i
zarządzanie ruchem miejskim.

\item {} 
\sphinxAtStartPar
Era Internetu Rzeczy (IoT) \textendash{} połączenie urządzeń codziennego użytku z
Internetem, umożliwiające zbieranie danych, inteligentne zarządzanie
środowiskiem fizycznym, monitorowanie infrastruktury miejskiej oraz
automatyzację gospodarstw domowych.

\item {} 
\sphinxAtStartPar
Era komputerów kwantowych i zaawansowanych algorytmów \textendash{} wprowadzenie
nowych paradygmatów obliczeniowych, które znacząco zwiększają moc
przetwarzania danych i rozwiązywania złożonych problemów, takich jak
modelowanie molekularne, optymalizacja logistyczna czy kryptografia
postkwantowa.
Każdy z tych etapów charakteryzował się nie tylko wzrostem możliwości
technicznych, ale także koniecznością adaptacji społecznej,
edukacyjnej i prawnej. W miarę rozwoju technologii powstają też nowe
modele regulacyjne, standardy i normy bezpieczeństwa, które mają
chronić użytkowników i organizacje.

\end{itemize}

\sphinxstepscope


\chapter{Znaczenie technologii informacyjnych w społeczeństwie}
\label{\detokenize{rozdzial4/index:znaczenie-technologii-informacyjnych-w-spoleczenstwie}}\label{\detokenize{rozdzial4/index::doc}}
\sphinxAtStartPar
Technologie informacyjne zmieniają sposób, w jaki funkcjonujemy na co
dzień. Przykładowe zastosowania to:
\begin{itemize}
\item {} 
\sphinxAtStartPar
edukacja zdalna i platformy e\sphinxhyphen{}learningowe, umożliwiające naukę na
odległość, współpracę między uczniami i nauczycielami, tworzenie
wirtualnych laboratoriów i zasobów edukacyjnych.

\item {} 
\sphinxAtStartPar
bankowość elektroniczna i płatności mobilne, przyspieszające
transakcje, zmniejszające ryzyko błędów, oferujące nowe metody
zarządzania finansami osobistymi i firmowymi.

\item {} 
\sphinxAtStartPar
systemy zarządzania przedsiębiorstwami (ERP, CRM), wspierające
planowanie zasobów, analizę danych, obsługę klienta, raportowanie i
przewidywanie trendów rynkowych.

\item {} 
\sphinxAtStartPar
telemedycyna i cyfrowe rejestracje pacjentów, pozwalające na zdalne
konsultacje, monitorowanie stanu zdrowia w czasie rzeczywistym,
analizę danych medycznych i rozwój inteligentnych systemów
diagnostycznych.

\item {} 
\sphinxAtStartPar
administracja elektroniczna, umożliwiająca składanie dokumentów
online, zarządzanie sprawami urzędowymi, przyspieszanie procesów
decyzyjnych oraz zwiększanie transparentności działania instytucji
publicznych.
Technologie informacyjne wprowadzają też nowe standardy w dziedzinie
komunikacji, zarządzania projektami i współpracy międzynarodowej.
Ułatwiają wymianę wiedzy, rozwój społeczności naukowych i branżowych,
a także przyczyniają się do szybkiego reagowania na kryzysy, takie
jak katastrofy naturalne czy zagrożenia zdrowotne.

\end{itemize}


\begin{savenotes}\sphinxattablestart
\sphinxthistablewithglobalstyle
\centering
\begin{tabulary}{\linewidth}[t]{TTTTT}
\sphinxtoprule
\sphinxstyletheadfamily 
\sphinxAtStartPar
Technologia
&\sphinxstyletheadfamily 
\sphinxAtStartPar
Zatosowanie
&\sphinxstyletheadfamily 
\sphinxAtStartPar
Zalety
&\sphinxstyletheadfamily 
\sphinxAtStartPar
Wyzwania
&\sphinxstyletheadfamily 
\sphinxAtStartPar
Przykłady praktyczne
\\
\sphinxmidrule
\sphinxtableatstartofbodyhook
\sphinxAtStartPar
Chmura
sieciowa
&
\sphinxAtStartPar
Przechowywanie
i przetwarzanie
danych
&
\sphinxAtStartPar
Skalowalność,
dostępność
zdalna
&
\sphinxAtStartPar
Bezpieczeństwo,
koszty
&
\sphinxAtStartPar
AWS, Google Cloud,
Microsoft Azure
\\
\sphinxhline
\sphinxAtStartPar
Sztuczna
inteligencja
&
\sphinxAtStartPar
Analiza danych,
automatyzacja
procesów
&
\sphinxAtStartPar
Predykcja
trendów
automatyzacja
&
\sphinxAtStartPar
Etyka, wymagana
jakość danych
&
\sphinxAtStartPar
Chatboty, systemy
rekomendacyjne
\\
\sphinxhline
\sphinxAtStartPar
Internet
rzeczy
&
\sphinxAtStartPar
Smart home,
inteligentne
miasta
&
\sphinxAtStartPar
Zdalne
monitorowanie,
oszczędzanie
energii
&
\sphinxAtStartPar
Prywatność
interoperacyjność
&
\sphinxAtStartPar
Smart home devices,
systemy monitoringu
miasta
\\
\sphinxhline
\sphinxAtStartPar
Blockchain
&
\sphinxAtStartPar
Kryptowaluty,
rejestry
transakcji
&
\sphinxAtStartPar
Niezmienność
danych,
transparentność
&
\sphinxAtStartPar
Skalowalność
zużycie energii
&
\sphinxAtStartPar
Bitcoin, Etherum,
rejestry łańcuchowe
\\
\sphinxhline
\sphinxAtStartPar
Telemedycyna
&
\sphinxAtStartPar
Zdalna opieka
medyczna
&
\sphinxAtStartPar
Wygoda
pacjenta,
monitorowanie
stanu zdrowia
nta,
&
\sphinxAtStartPar
Prywatność
pacjentów
danych,
wymogi
regulacyjne
&
\sphinxAtStartPar
Konsultacje online,
monitorowanie
\\
\sphinxbottomrule
\end{tabulary}
\sphinxtableafterendhook\par
\sphinxattableend\end{savenotes}

\sphinxstepscope


\chapter{Przyszłość technologii informacyjnych}
\label{\detokenize{rozdzial5/index:przyszlosc-technologii-informacyjnych}}\label{\detokenize{rozdzial5/index::doc}}
\sphinxAtStartPar
Przyszłość technologii informacyjnych wiąże się z coraz głębszą
integracją z codziennym życiem. Rozwój takich dziedzin jak Internet
Rzeczy, sztuczna inteligencja, blockchain, rzeczywistość rozszerzona,
komputery kwantowe, automatyzacja przemysłowa i robotyka może
całkowicie odmienić sposób, w jaki przetwarzamy dane, podejmujemy
decyzje i komunikujemy się.
Pojawiają się też nowe wyzwania:
\begin{itemize}
\item {} 
\sphinxAtStartPar
Prywatność danych \textendash{} ochrona informacji osobistych i firmowych w
świecie cyfrowym, tworzenie regulacji prawnych i standardów ochrony
danych.

\item {} 
\sphinxAtStartPar
Bezpieczeństwo cybernetyczne \textendash{} zabezpieczenie systemów przed coraz
bardziej zaawansowanymi cyberatakami, rozwój technologii
antywłamaniowych i monitoringu sieciowego.

\item {} 
\sphinxAtStartPar
Etyka technologiczna \textendash{} odpowiedzialne wdrażanie sztucznej
inteligencji, algorytmów decyzyjnych i systemów automatyzujących pracę
ludzi.

\item {} 
\sphinxAtStartPar
Zrównoważony rozwój \textendash{} minimalizacja wpływu infrastruktury
informatycznej na środowisko naturalne, projektowanie
energooszczędnych systemów i recykling elektroniki.

\item {} 
\sphinxAtStartPar
Edukacja i adaptacja społeczna \textendash{} konieczność uczenia nowych
kompetencji w świecie dynamicznie zmieniającej się technologii, rozwój
szkoleń, kursów online i programów certyfikacyjnych.

\item {} 
\sphinxAtStartPar
Rozwój technologii informacyjnych nie ogranicza się jedynie do
sektora biznesowego; znajduje zastosowanie w edukacji, medycynie,
logistyce, administracji publicznej, badaniach naukowych oraz w
codziennym życiu każdego człowieka. Przewiduje się, że w nadchodzących
latach technologia informacyjna stanie się jeszcze bardziej
nieodłącznym elementem codzienności, a jej znaczenie będzie rosło w
każdej dziedzinie życia społecznego i gospodarczego.

\end{itemize}

\sphinxstepscope


\chapter{Podsumowanie}
\label{\detokenize{rozdzial6/index:podsumowanie}}\label{\detokenize{rozdzial6/index::doc}}
\sphinxAtStartPar
Technologie informacyjne stanowią fundament nowoczesnego społeczeństwa.
Dzięki nim możliwy jest szybki przepływ danych, globalna komunikacja,
automatyzacja procesów, rozwój gospodarczy i naukowy, a także tworzenie
inteligentnych systemów wspierających życie ludzi. Ich dalszy rozwój
będzie miał ogromny wpływ na przyszłość, dlatego zrozumienie zasad ich
działania i umiejętność wykorzystania technologii informacyjnych to dziś
kluczowa kompetencja. W miarę postępu technologicznego wzrasta też
potrzeba odpowiedzialnego i świadomego podejścia do korzystania z
narzędzi cyfrowych, aby zapewnić bezpieczeństwo, efektywność, równowagę
społeczną i zrównoważony rozwój w społeczeństwie. Technologie
informacyjne będą nadal napędzać innowacje, zmieniać rynek pracy,
edukację i życie codzienne, tworząc jednocześnie nowe możliwości i
wyzwania dla ludzi i instytucji.



\renewcommand{\indexname}{Indeks}
\printindex
\end{document}